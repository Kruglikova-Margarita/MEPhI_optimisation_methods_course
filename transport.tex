\chapter{Транспортная задача}

Решить транспортную задачу с m = 3 поставщиками и n = 7 потребителями. Данные о запасах, спросе и стоимости транспортировки приведены в таблице.

\begin{center}
    \begin{tabular}{|a | c | c | c | c | c | c | c | c|} 
         \hline
         \rowcolor{LightGray}
            & 1 & 2 & 3 & 4 & 5 & 6 & 7 & запасы\\
         \hline
            1 & 5 & 8 & 1 & 4 & 2 & 7 & 3 & 150\\
         \hline
            2 & 3 & 6 & 4 & 9 & 1 & 8 & 2 & 100\\
         \hline
            3 & 7 & 1 & 5 & 8 & 3 & 2 & 4 & 150\\
         \hline
            спрос & 60 & 40 & 70 & 50 & 90 & 40 & 50 & $\sum =$ 400\\
        \hline
    \end{tabular}
\end{center}

\begin{center}
    {\bf
    Решение:}
\end{center}

$5x_{11} + 8x_{12} + x_{13} + 4x_{14} + 2x_{15} + 7x_{16} + 3x_{17} + 3x_{21} + 6x_{22} + 4x_{23} + 9x_{24} +\\+ x_{25} + 8x_{26} + 2x_{27} + 7x_{31} + x_{32} + 5x_{33} + 8x_{34} + 3x_{35} + 2x_{36} + 4x_{37} \rightarrow min$

\begin{equation*}
    \begin{cases}
        x_{11} + x_{12} + x_{13} + x_{14} + x_{15} + x_{16} + x_{17} = 150\\
        x_{21} + x_{22} + x_{23} + x_{24} + x_{25} + x_{26} + x_{27} = 100\\
        x_{31} + x_{32} + x_{33} + x_{34} + x_{35} + x_{36} + x_{37} = 150\\
        x_{11} + x_{21} + x_{31} = 60\\
        x_{12} + x_{22} + x_{32} = 40\\
        x_{13} + x_{23} + x_{33} = 70\\
        x_{14} + x_{24} + x_{34} = 50\\
        x_{15} + x_{25} + x_{35} = 90\\
        x_{16} + x_{26} + x_{36} = 40\\
        x_{17} + x_{27} + x_{37} = 50\\
        x_{ij} \ge 0, i = \overline{1, 3}, j = \overline{1, 7} 
    \end{cases}
\end{equation*}

\begin{flushleft}
{\bf1) Проверим необходимое и достаточное условие разрешимости задачи:}\\
Сумма запасов $= 150 + 100 + 150 = 400$\\
Сумма спросов $= 60 + 40 + 70 + 50 + 90 + 40 + 50 = 400$\\
Условие баланса соблюдается. Запасы равны потребностям. Следовательно, модель транспортной задачи является закрытой.\\

{\bf2) Найдем начальное опорное решение}\\
{\bf2.1) Через метод северо-западного угла:}
\end{flushleft}

\begin{center}
    \begin{tabular}{|a | c | c | c | c | c | c | c | c|} 
         \hline
         \rowcolor{LightGray}
            & 1 & 2 & 3 & 4 & 5 & 6 & 7 & запасы\\
         \hline
            1 & \cellcolor{Beige} $5^{60}$ & \cellcolor{Beige} $8^{40}$ & \cellcolor{Beige} $1^{50}$ & 4 & 2 & 7 & 3 & 150\\
         \hline
            2 & 3 & 6 & \cellcolor{Beige} $4^{20}$ & \cellcolor{Beige} $9^{50}$ & \cellcolor{Beige} $1^{30}$ & 8 & 2 & 100\\
         \hline
            3 & 7 & 1 & 5 & 8 & \cellcolor{Beige} $3^{60}$ & \cellcolor{Beige} $2^{40}$ & \cellcolor{Beige} $4^{50}$ & 150\\
         \hline
            спрос & 60 & 40 & 70 & 50 & 90 & 40 & 50 & $\sum =$ 400\\
        \hline
    \end{tabular}
\end{center}

\begin{flushleft}
Число закрашенных клеток: 9\\
$m + n - 1 = 3 + 7 - 1 = 9$\\
Значит опорный план невырожденный\\
Значение целевой функции:\\
$F(x) = 5 \cdot 60 + 8 \cdot 40 + 1 \cdot 50 + 4 \cdot 20 + 9 \cdot 50 + 1 \cdot 30 + 3 \cdot 60 + 2 \cdot 40 + 4 \cdot 50 = 1690$

{\bf2.2) Через метод минимальных элементов:}
\end{flushleft}

\begin{center}
    \begin{tabular}{|a | c | c | c | c | c | c | c | c|} 
         \hline
         \rowcolor{LightGray}
            & 1 & 2 & 3 & 4 & 5 & 6 & 7 & запасы\\
         \hline
            1 & 5 & 8 & \cellcolor{Beige} $1^{70}$ & \cellcolor{Beige} $4^{40}$ & 2 & 7 & \cellcolor{Beige} $3^{40}$ & 150\\
         \hline
            2 & 3 & 6 & 4 & 9 & \cellcolor{Beige} $1^{90}$ & 8 & \cellcolor{Beige} $2^{10}$ & 100\\
         \hline
            3 & \cellcolor{Beige} $7^{60}$ & \cellcolor{Beige} $1^{40}$ & 5 & \cellcolor{Beige} $8^{10}$ & 3 & \cellcolor{Beige} $2^{40}$ & 4 & 150\\
         \hline
            спрос & 60 & 40 & 70 & 50 & 90 & 40 & 50 & $\sum =$ 400\\
        \hline
    \end{tabular}
\end{center}

\begin{flushleft}
Число закрашенных клеток: 9\\
$m + n - 1 = 3 + 7 - 1 = 9$\\
Значит опорный план невырожденный\\
Значение целевой функции:\\
$F(x) = 7 \cdot 60 + 1 \cdot 40 + 1 \cdot 70 + 4 \cdot 40 + 8 \cdot 10 + 1 \cdot 90 + 2 \cdot 40 + 3 \cdot 40 + 2 \cdot 10 = 1080$
{\bf3) Решим задачу с помощью методов потенциалов}\\
В качестве опорного решение возьмем решение, полученное методом северно-западного угла.
\end{flushleft}

\begin{center}
    \begin{tabular}{|a | c | c | c | c | c | c | c | c|} 
         \hline
         \rowcolor{LightGray}
            & 1 & 2 & 3 & 4 & 5 & 6 & 7 & $v_{i}$\\
         \hline
            1 & \cellcolor{Beige} $5^{60}$ & \cellcolor{Beige} $8^{40}$ & \cellcolor{Beige} $1^{50}$ & 4 & 2 & 7 & 3 & $v_1 = 5$\\
         \hline
            2 & 3 & 6 & \cellcolor{Beige} $4^{20}$ & \cellcolor{Beige} $9^{50}$ & \cellcolor{Beige} $1^{30}$ & 8 & 2 & $v_2 = 8$\\
         \hline
            3 & 7 & 1 & 5 & 8 & \cellcolor{Beige} $3^{60}$ & \cellcolor{Beige} $2^{40}$ & \cellcolor{Beige} $4^{50}$ & $v_3 = 10$\\
         \hline
            $u_{i}$ & $u_1 = 0$ & $u_2 = 3$ & $u_3 = -4$ & $u_4 = 1$ & $u_5 = -7$ & $u_6 = -8$ & $u_7 = -6$ & \\
        \hline
    \end{tabular}
\end{center}

\begin{flushleft}
Рассчитаем $d_{ij} = c_{ij} - u_{j} - v_{i}$\\
$d_{14} = 4 - 1 - 5 = -2$\\
$d_{15} = 2 - (-7) - 5 = 4$\\
$d_{16} = 7 - (-8) - 5 = 10$\\
$d_{17} = 3 - (-6) - 5 = 4$\\
$d_{21} = 3 - 0 - 8 = -5$\\
$d_{22} = 6 - 3 - 8 = -5$\\
$d_{26} = 8 - (-8) - 8 = 8$\\
$d_{27} = 2 - (-6) - 8 = 0$\\
$d_{31} = 7 - 0 - 10 = -3$\\
$d_{32} = 1 - 3 - 10 = -12$\\
$d_{33} = 5 - (-4) - 10 = -1$\\
$d_{34} = 8 - 1 - 10 = -3$\\
Не все $d_{ij} \ge 0 \implies$ решение не оптимально.\\
Минимальное $d_{ij} = d_{32} = -12 \implies$ клетка пересчета: (3; 2)\\
Цикл пересчета:\\
$(3; 2) \rightarrow (3; 3) \rightarrow (3; 4) \rightarrow (3; 5) \rightarrow (2; 5) \rightarrow (2; 4) \rightarrow (2; 3) \rightarrow (1; 3) \rightarrow (1; 2) \rightarrow (2; 2) \rightarrow (3; 2)$
\end{flushleft}

\begin{center}
    \begin{tabular}{|a | c | c | c | c | c | c | c | c|} 
         \hline
         \rowcolor{LightGray}
            & 1 & 2 & 3 & 4 & 5 & 6 & 7 & $v_{i}$\\
         \hline
            1 & \cellcolor{Beige} $5^{60}$ & \cellcolor{Beige} $8^{40}$ {\bf-} & \cellcolor{Beige} $1^{50}$ {\bf+} & 4 & 2 & 7 & 3 & $v_1 = 5$\\
         \hline
            2 & 3 & 6 & \cellcolor{Beige} $4^{20}$ {\bf-} & \cellcolor{Beige} $9^{50}$ & \cellcolor{Beige} $1^{30}$ {\bf+} & 8 & 2 & $v_2 = 8$\\
         \hline
            3 & 7 & 1 {\bf+} & 5 & 8 & \cellcolor{Beige} $3^{60}$ {\bf-} & \cellcolor{Beige} $2^{40}$ & \cellcolor{Beige} $4^{50}$ & $v_3 = 10$\\
         \hline
            $u_{i}$ & $u_1 = 0$ & $u_2 = 3$ & $u_3 = -4$ & $u_4 = 1$ & $u_5 = -7$ & $u_6 = -8$ & $u_7 = -6$ & \\
        \hline
    \end{tabular}
\end{center}

Новый опорный план:
\begin{center}
    \begin{tabular}{|a | c | c | c | c | c | c | c | c|} 
         \hline
         \rowcolor{LightGray}
            & 1 & 2 & 3 & 4 & 5 & 6 & 7 & $v_{i}$\\
         \hline
            1 & \cellcolor{Beige} $5^{60}$ & \cellcolor{Beige} $8^{20}$ & \cellcolor{Beige} $1^{70}$ & 4 & 2 & 7 & 3 & $v_1 = 5$\\
         \hline
            2 & 3 & 6 & 4 & \cellcolor{Beige} $9^{50}$ & \cellcolor{Beige} $1^{50}$ & 8 & 2 & $v_2 = -4$\\
         \hline
            3 & 7 & \cellcolor{Beige} $1^{20}$ & 5 & 8 & \cellcolor{Beige} $3^{40}$ & \cellcolor{Beige} $2^{40}$ & \cellcolor{Beige} $4^{50}$ & $v_3 = -2$\\
         \hline
            $u_{i}$ & $u_1 = 0$ & $u_2 = 3$ & $u_3 = -4$ & $u_4 = 13$ & $u_5 = 5$ & $u_6 = 4$ & $u_7 = 6$ & \\
        \hline
    \end{tabular}
\end{center}

\begin{flushleft}
Число закрашенных клеток: 9\\
$m + n - 1 = 3 + 7 - 1 = 9$\\
Значит опорный план невырожденный\\
Проверим на оптимальность:\\
$d_{14} = 4 - 13 - 5 = -14$\\
$d_{15} = 2 - 5 - 5 = -8$\\
$d_{16} = 7 - 4 - 5 = -2$\\
$d_{17} = 3 - 6 - 5 = -8$\\
$d_{21} = 3 - 0 - (-4) = 7$\\
$d_{22} = 6 - 3 - (-4) = 7$\\
$d_{23} = 4 - (-4) - (-4) = 12$\\
$d_{26} = 8 - 4 - (-4) = 8$\\
$d_{27} = 2 - 6 - (-4) = 0$\\
$d_{31} = 7 - 0 - (-2) = 9$\\
$d_{33} = 5 - (-4) - (-2) = 11$\\
$d_{34} = 8 - 13 - (-2) = -3$\\
Не все $d_{ij} \ge 0 \implies$ решение не оптимально.\\
Минимальное $d_{ij} = d_{14} = -14 \implies$ клетка пересчета: (1; 4)\\
Цикл пересчета:\\
$(1; 4) \rightarrow (1; 3) \rightarrow (1; 2) \rightarrow (2; 2) \rightarrow (3; 2) \rightarrow (3; 3) \rightarrow (3; 4) \rightarrow (3; 5) \rightarrow (2; 5) \rightarrow (2; 4) \rightarrow (1; 4)$
\end{flushleft}

\begin{center}
    \begin{tabular}{|a | c | c | c | c | c | c | c | c|} 
         \hline
         \rowcolor{LightGray}
            & 1 & 2 & 3 & 4 & 5 & 6 & 7 & $v_{i}$\\
         \hline
            1 & \cellcolor{Beige} $5^{60}$ & \cellcolor{Beige} $8^{20}$ {\bf-} & \cellcolor{Beige} $1^{70}$ & 4 {\bf+} & 2 & 7 & 3 & $v_1 = 5$\\
         \hline
            2 & 3 & 6 & 4 & \cellcolor{Beige} $9^{50}$ {\bf-} & \cellcolor{Beige} $1^{50}$ {\bf+} & 8 & 2 & $v_2 = -4$\\
         \hline
            3 & 7 & \cellcolor{Beige} $1^{20}$ {\bf+} & 5 & 8 & \cellcolor{Beige} $3^{40}$ {\bf-} & \cellcolor{Beige} $2^{40}$ & \cellcolor{Beige} $4^{50}$ & $v_3 = -2$\\
         \hline
            $u_{i}$ & $u_1 = 0$ & $u_2 = 3$ & $u_3 = -4$ & $u_4 = 13$ & $u_5 = 5$ & $u_6 = 4$ & $u_7 = 6$ & \\
        \hline
    \end{tabular}
\end{center}

Новый опорный план:
\begin{center}
    \begin{tabular}{|a | c | c | c | c | c | c | c | c|} 
         \hline
         \rowcolor{LightGray}
            & 1 & 2 & 3 & 4 & 5 & 6 & 7 & $v_{i}$\\
         \hline
            1 & \cellcolor{Beige} $5^{60}$ & 8 & \cellcolor{Beige} $1^{70}$ & \cellcolor{Beige} $4^{20}$ & 2 & 7 & 3 & {\small$v_1 = 5$}\\
         \hline
            2 & 3 & 6 & 4 & \cellcolor{Beige} $9^{30}$ & \cellcolor{Beige} $1^{70}$ & 8 & 2 & {\small$v_2 = 10$}\\
         \hline
            3 & 7 & \cellcolor{Beige} $1^{40}$ & 5 & 8 & \cellcolor{Beige} $3^{20}$ & \cellcolor{Beige} $2^{40}$ & \cellcolor{Beige} $4^{50}$ & {\small$v_3 = 12$}\\
         \hline
            $u_{i}$ & {\small$u_1 = 0$} & {\small$u_2 = -11$} & {\small$u_3 = -4$} & {\small$u_4 = -1$} & {\small$u_5 = -9$} & {\small$u_6 = -10$} & {\small$u_7 = -8$} & \\
        \hline
    \end{tabular}
\end{center}

\begin{flushleft}
Число закрашенных клеток: 9\\
$m + n - 1 = 3 + 7 - 1 = 9$\\
Значит опорный план невырожденный\\
Проверим на оптимальность:\\
$d_{12} = 8 - (-11) - 5 = 14$\\
$d_{15} = 2 - (-9) - 5 = 6$\\
$d_{16} = 7 - (-10) - 5 = 12$\\
$d_{17} = 3 - (-8) - 5 = 6$\\
$d_{21} = 3 - 0 - 10 = -7$\\
$d_{22} = 6 - (-11) - 10 = 7$\\
$d_{23} = 4 - (-4) - 10 = -2$\\
$d_{26} = 8 - (-10) - 10 = 8$\\
$d_{27} = 2 - (-8) - 10 = 0$\\
$d_{31} = 7 - 0 - 12 = -5$\\
$d_{33} = 5 - (-4) - 12 = -3$\\
$d_{34} = 8 - (-1) - 12 = -3$\\
Не все $d_{ij} \ge 0 \implies$ решение не оптимально.\\
Минимальное $d_{ij} = d_{21} = -7 \implies$ клетка пересчета: (2; 1)\\
Цикл пересчета:\\
$(2; 1) \rightarrow (1; 1) \rightarrow (1; 2) \rightarrow (1; 3) \rightarrow (1; 4) \rightarrow (2; 4) \rightarrow (2; 3) \rightarrow (2; 2) \rightarrow (2; 1)$
\end{flushleft}

\begin{center}
    \begin{tabular}{|a | c | c | c | c | c | c | c | c|} 
         \hline
         \rowcolor{LightGray}
            & 1 & 2 & 3 & 4 & 5 & 6 & 7 & $v_{i}$\\
         \hline
            1 & \cellcolor{Beige} $5^{60}$ {\bf-} & 8 & \cellcolor{Beige} $1^{70}$ & \cellcolor{Beige} $4^{20}$ {\bf+} & 2 & 7 & 3 & {\small$v_1 = 5$}\\
         \hline
            2 & 3 {\bf+} & 6 & 4 & \cellcolor{Beige} $9^{30}$ {\bf-} & \cellcolor{Beige} $1^{70}$ & 8 & 2 & {\small$v_2 = 10$}\\
         \hline
            3 & 7 & \cellcolor{Beige} $1^{40}$ & 5 & 8 & \cellcolor{Beige} $3^{20}$ & \cellcolor{Beige} $2^{40}$ & \cellcolor{Beige} $4^{50}$ & {\small$v_3 = 12$}\\
         \hline
            $u_{i}$ & {\small$u_1 = 0$} & {\small$u_2 = -11$} & {\small$u_3 = -4$} & {\small$u_4 = -1$} & {\small$u_5 = -9$} & {\small$u_6 = -10$} & {\small$u_7 = -8$} & \\
        \hline
    \end{tabular}
\end{center}

Новый опорный план:
\begin{center}
    \begin{tabular}{|a | c | c | c | c | c | c | c | c|} 
         \hline
         \rowcolor{LightGray}
            & 1 & 2 & 3 & 4 & 5 & 6 & 7 & $v_{i}$\\
         \hline
            1 & \cellcolor{Beige} $5^{30}$ & 8 & \cellcolor{Beige} $1^{70}$ & \cellcolor{Beige} $4^{50}$ & 2 & 7 & 3 & {\small$v_1 = 5$}\\
         \hline
            2 & \cellcolor{Beige} $3^{30}$ & 6 & 4 & 9 & \cellcolor{Beige} $1^{70}$ & 8 & 2 & {\small$v_2 = 3$}\\
         \hline
            3 & 7 & \cellcolor{Beige} $1^{40}$ & 5 & 8 & \cellcolor{Beige} $3^{20}$ & \cellcolor{Beige} $2^{40}$ & \cellcolor{Beige} $4^{50}$ & {\small$v_3 = 5$}\\
         \hline
            $u_{i}$ & {\small$u_1 = 0$} & {\small$u_2 = -4$} & {\small$u_3 = -4$} & {\small$u_4 = -1$} & {\small$u_5 = -2$} & {\small$u_6 = -3$} & {\small$u_7 = -1$} & \\
        \hline
    \end{tabular}
\end{center}

\begin{flushleft}
Число закрашенных клеток: 9\\
$m + n - 1 = 3 + 7 - 1 = 9$\\
Значит опорный план невырожденный\\
Проверим на оптимальность:\\
$d_{12} = 8 - (-4) - 5 = 7$\\
$d_{15} = 2 - (-2) - 5 = -1$\\
$d_{16} = 7 - (-3) - 5 = 5$\\
$d_{17} = 3 - (-1) - 5 = -1$\\
$d_{22} = 6 - (-4) - 3 = 7$\\
$d_{23} = 4 - (-4) - 3 = 5$\\
$d_{24} = 9 - (-1) - 3 = 7$\\
$d_{26} = 8 - (-3) - 3 = 8$\\
$d_{27} = 2 - (-1) - 3 = 0$\\
$d_{31} = 7 - 0 - 5 = 2$\\
$d_{33} = 5 - (-4) - 5 = 4$\\
$d_{34} = 8 - (-1) - 5 = 4$\\
Не все $d_{ij} \ge 0 \implies$ решение не оптимально.\\
Минимальное $d_{ij} = d_{15} = -1 \implies$ клетка пересчета: (1; 5)\\
Цикл пересчета:\\
$(1; 5) \rightarrow (2; 5) \rightarrow (2; 4) \rightarrow (2; 3) \rightarrow (2; 2) \rightarrow (2; 1) \rightarrow (1; 1) \rightarrow (1; 2) \rightarrow (1; 3) \rightarrow (1; 4) \rightarrow (1; 5)$
\end{flushleft}

\begin{center}
    \begin{tabular}{|a | c | c | c | c | c | c | c | c|} 
         \hline
         \rowcolor{LightGray}
            & 1 & 2 & 3 & 4 & 5 & 6 & 7 & $v_{i}$\\
         \hline
            1 & \cellcolor{Beige} $5^{30}$ {\bf-}  & 8 & \cellcolor{Beige} $1^{70}$ & \cellcolor{Beige} $4^{50}$ & 2 {\bf+} & 7 & 3 & {\small$v_1 = 5$}\\
         \hline
            2 & \cellcolor{Beige} $3^{30}$ {\bf+}  & 6 & 4 & 9 & \cellcolor{Beige} $1^{70}$ {\bf-} & 8 & 2 & {\small$v_2 = 3$}\\
         \hline
            3 & 7 & \cellcolor{Beige} $1^{40}$ & 5 & 8 & \cellcolor{Beige} $3^{20}$ & \cellcolor{Beige} $2^{40}$ & \cellcolor{Beige} $4^{50}$ & {\small$v_3 = 5$}\\
         \hline
            $u_{i}$ & {\small$u_1 = 0$} & {\small$u_2 = -4$} & {\small$u_3 = -4$} & {\small$u_4 = -1$} & {\small$u_5 = -2$} & {\small$u_6 = -3$} & {\small$u_7 = -1$} & \\
        \hline
    \end{tabular}
\end{center}

Новый опорный план:
\begin{center}
    \begin{tabular}{|a | c | c | c | c | c | c | c | c|} 
         \hline
         \rowcolor{LightGray}
            & 1 & 2 & 3 & 4 & 5 & 6 & 7 & $v_{i}$\\
         \hline
            1 & 5  & 8 & \cellcolor{Beige} $1^{70}$ & \cellcolor{Beige} $4^{50}$ & \cellcolor{Beige} $2^{30}$ & 7 & 3 & {\small$v_1 = 4$}\\
         \hline
            2 & \cellcolor{Beige} $3^{60}$ & 6 & 4 & 9 & \cellcolor{Beige} $1^{40}$ & 8 & 2 & {\small$v_2 = 3$}\\
         \hline
            3 & 7 & \cellcolor{Beige} $1^{40}$ & 5 & 8 & \cellcolor{Beige} $3^{20}$ & \cellcolor{Beige} $2^{40}$ & \cellcolor{Beige} $4^{50}$ & {\small$v_3 = 5$}\\
         \hline
            $u_{i}$ & {\small$u_1 = 0$} & {\small$u_2 = -4$} & {\small$u_3 = -3$} & {\small$u_4 = 0$} & {\small$u_5 = -2$} & {\small$u_6 = -3$} & {\small$u_7 = -1$} & \\
        \hline
    \end{tabular}
\end{center}

\begin{flushleft}
Число закрашенных клеток: 9\\
$m + n - 1 = 3 + 7 - 1 = 9$\\
Значит опорный план невырожденный\\
Проверим на оптимальность:\\
$d_{11} = 5 - 0 - 4 = 1$\\
$d_{12} = 8 - (-4) - 4 = 8$\\
$d_{16} = 7 - (-3) - 4 = 6$\\
$d_{17} = 3 - (-1) - 4 = 0$\\
$d_{22} = 6 - (-4) - 3 = 7$\\
$d_{23} = 4 - (-3) - 3 = 4$\\
$d_{24} = 9 - 0 - 3 = 6$\\
$d_{26} = 8 - (-3) - 3 = 8$\\
$d_{27} = 2 - (-1) - 3 = 0$\\
$d_{31} = 7 - 0 - 5 = 2$\\
$d_{33} = 5 - (-3) - 5 = 3$\\
$d_{34} = 8 - 0 - 5 = 3$\\
Все $d_{ij} \ge 0 \implies$ решение оптимально.\\
{\bfЗначение целевой функции:\\
$F(x) = 1 \cdot 70 + 4 \cdot 50 + 2 \cdot 30 + 3 \cdot 60 + 1 \cdot 40 + 1 \cdot 40 + 3 \cdot 20 + 2 \cdot 40 + 4 \cdot 50 = 930$}
\end{flushleft}

\newpage

{\bfКод программы на Python для решения транспортной задачи (метод минимального элемента)}

\begin{lstlisting}
import numpy as np
from scipy.optimize import linprog

def transportation_problem(supplies, demands, costs):
    if sum(supplies) != sum(demands):
        raise ValueError("Sum of demands must be equal to sum of supplies")

    supplies = np.array(supplies)
    demands = np.array(demands)
    costs = np.array(costs)

    m, n = len(supplies), len(demands)
    c = costs.flatten()
    A_eq = []
    for i in range(m):
        row = [0] * (m * n)
        for j in range(n):
            row[i * n + j] = 1
        A_eq.append(row)

    for j in range(n):
        row = [0] * (m * n)
        for i in range(m):
            row[i * n + j] = 1
        A_eq.append(row)

    A_eq = np.array(A_eq)
    b_eq = np.concatenate([supplies, demands])

    bounds = [(0, None) for _ in range(m * n)]

    result = linprog(c, A_eq=A_eq, b_eq=b_eq, bounds=bounds, method="highs")

    if not result.success:
        raise ValueError("No solution was found: " + result.message)

    allocation = result.x.reshape((m, n))
    return allocation

supplies = [150, 100, 150]
demands = [60, 40, 70, 50, 90, 40, 50]
costs = [
    [5, 8, 1, 4, 2, 7, 3],
    [3, 6, 4, 9, 1, 8, 2],
    [7, 1, 5, 8, 3, 2, 4]
]

allocation = transportation_problem(supplies, demands, costs)
print("Plan:")
print(np.round(allocation).astype(int))

print(sum(sum(allocation*np.array(costs))))
\end{lstlisting}

\newpage

\begin{figure}[h]
\centering
\includegraphics[]{тз.png}
\centering
\caption{Результат работы программы}
\end{figure}
