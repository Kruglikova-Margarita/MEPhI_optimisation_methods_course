\chapter{Симплекс метод}

{\bf Вариант №1\\Найти решение задачи линейного программирования симплекс – методом (для "а, с" на max, для "в" на max и min)}

\subsubsection{a)}
\begin{equation*}
    F = x_1 + x_2 \rightarrow max
\end{equation*}
\begin{equation*}
    \begin{cases}
        x_1 + 2x_2 \le 8 \\
        6x_1 - x_2 \le 3 \\
        x_1 \ge 0, x_2 \ge 0
    \end{cases}
\end{equation*}

\begin{center}
    {\bf
    Решение:}
\end{center}

\begin{flushleft}
Приведем задачу к канонической форме:
\end{flushleft}

\begin{equation*}
    \begin{cases}
        x_1 + 2x_2 + x_3 = 8 \\
        6x_1 - x_2 + x_4 = 3 \\
        x_i \ge 0, i = 1, 2, 3, 4
    \end{cases}
\end{equation*}
\begin{center}
    $F - x_1 - x_2 = 0$
\end{center}

\begin{flushleft}
    {\bf1-я симплекс-таблица:}\\
\end{flushleft}

\begin{center}
    \begin{tabular}{|c | c g c c c c|} 
         \hline
            Базис & $x_1$ & $x_2$ & $x_3$ & $x_4$ & $b_i$ & $b_i/$р.с.\\
         \hline
         \rowcolor{LightBlue}
            $x_3$ & 1 & \cellcolor{cyan}2 & 1 & 0 & 8 & 8/2 = 4\\
         \hline
            $x_4$ & 6 & -1 & 0 & 1 & 3 & 3/(-1) = -3\\
         \hline
            F(x) & -1 & -1 & 0 & 0 & 0 &\\
         \hline
    \end{tabular}
\end{center}

\begin{flushleft}
    Наименьшее значение в строке F(x): -1\\
    Разрешающий столбец: $x_2$\\
    Минимальное положительное значение из столбца $b_i$/р.с. : 4\\
    Разрешающий элемент: 2\\
    Не все значения в строке F(x) положительные $\implies$ решение не оптимально, строим новую таблицу\\
    {\bf2-я симплекс-таблица:}\\
\end{flushleft}

\begin{center}
    \begin{tabular}{|c | g c c c c c|} 
         \hline
            Базис & $x_1$ & $x_2$ & $x_3$ & $x_4$ & $b_i$ & $b_i/$р.с.\\
         \hline
            $x_2$ & 1 & 2 & 1 & 0 & 8 & 8/1 = 8\\
         \hline
         \rowcolor{LightBlue}
            $x_4$ & \cellcolor{cyan}6.5 & 0 & 0.5 & 1 & 7 & 7/6.5 = $\frac{14}{13}$\\
         \hline
            F(x) & -0.5 & 0 & 0.5 & 0 & 4 &\\
         \hline
    \end{tabular}
\end{center}

\begin{flushleft}
    Наименьшее значение в строке F(x): -0.5\\
    Разрешающий столбец: $x_1$\\
    Минимальное положительное значение из столбца $b_i$/р.с. : $\frac{14}{13}$\\
    Разрешающий элемент: 6.5\\
    Не все значения в строке F(x) положительные $\implies$ решение не оптимально, строим новую таблицу\\
    {\bf3-я симплекс-таблица:}
\end{flushleft}

\begin{center}
    \begin{tabular}{|c | c c c c c c|} 
         \hline
            Базис & $x_1$ & $x_2$ & $x_3$ & $x_4$ & $b_i$ & $b_i/$р.с.\\
         \hline
            $x_2$ & 0 & 2 & $\frac{12}{13}$ & $-\frac{2}{13}$ & $\frac{90}{13}$ & \\
         \hline
            $x_1$ & 6.5 & 0 & 0.5 & 1 & 7 &\\
         \hline
            F(x) & 0 & 0 & $\frac{7}{13}$ & 1 & $\frac{59}{13}$ &\\
         \hline
    \end{tabular}
\end{center}

\begin{flushleft}
    Все значения в строке F(x) положительные, $F_{max}(x) = \frac{59}{13}$\\
    Оптимальное решение:\\
    $x_1 = 7:\frac{13}{2} = \frac{14}{13}$\\
    $x_2 = \frac{90}{13}:2 = \frac{45}{13}$
\end{flushleft}

{\bfОтвет:~} $F_{max} = F(\frac{14}{13}; \frac{45}{13}) = \frac{59}{13}$



\subsubsection{b)}
\begin{equation*}
    F = x_1 + 2x_2 \rightarrow max(min)
\end{equation*}
\begin{equation*}
    \begin{cases}
        x_1 + x_2 \ge 6 \\
        5x_1 - 10x_2 \le 10 \\
        x_1 \ge 0, x_2 \ge 0
    \end{cases}
\end{equation*}

\begin{center}
    {\bf
    Решение:}
\end{center}

\begin{flushleft}
Приведем задачу к канонической форме:
\end{flushleft}

\begin{equation*}
    \begin{cases}
        x_1 + x_2 - x_3 = 6 \\
        5x_1 - 10x_2 + x_4 = 10 \\
        x_i \ge 0, i = 1, 2, 3, 4
    \end{cases}
\end{equation*}
\begin{center}
    $F - x_1 - 2x_2 = 0$
\end{center}

\begin{flushleft}
    {\bf1. Задача на максимум}\\
    {\bf1-я симплекс-таблица:}\\
\end{flushleft}

\begin{center}
    \begin{tabular}{|c | c g c c c c|} 
         \hline
            Базис & $x_1$ & $x_2$ & $x_3$ & $x_4$ & $b_i$ & $b_i/$р.с.\\
         \hline
         \rowcolor{LightBlue}
            $x_3$ & 1 & \cellcolor{cyan}1 & -1 & 0 & 6 & 6/1 = 6\\
         \hline
            $x_4$ & 5 & -10 & 0 & 1 & 10 & 10/(-10) = -1\\
         \hline
            F(x) & -1 & -2 & 0 & 0 & 0 &\\
         \hline
    \end{tabular}
\end{center}

\begin{flushleft}
    Наименьшее значение в строке F(x): -2\\
    Разрешающий столбец: $x_2$\\
    Минимальное положительное значение из столбца $b_i$/р.с. : 6\\
    Разрешающий элемент: 1\\
    Не все значения в строке F(x) положительные $\implies$ решение не оптимально, строим новую таблицу\\
    {\bf2-я симплекс-таблица:}\\
\end{flushleft}

\begin{center}
    \begin{tabular}{|c | c c g c c c|} 
         \hline
            Базис & $x_1$ & $x_2$ & $x_3$ & $x_4$ & $b_i$ & $b_i/$р.с.\\
         \hline
            $x_2$ & 1 & 1 & -1 & 0 & 6 & 6/(-1) = -6\\
         \hline
            $x_4$ & 15 & 0 & -10 & 1 & 70 & 70/(-10) = -7\\
         \hline
            F(x) & 1 & 0 & -2 & 0 & 12 &\\
         \hline
    \end{tabular}
\end{center}

\begin{flushleft}
    Наименьшее значение в строке F(x): -2\\
    Разрешающий столбец: $x_3$\\
    Все значения в столбце $b_i$/р.с. отрицательные $\implies$ $F_{max}$ не определено
\end{flushleft}

\begin{flushleft}
    {\bf2. Задача на минимум}\\
\end{flushleft}

\begin{flushleft}
    В 1-й симплекс-таблице для задачи на максимум все значения в строке F(x) отрицательные $\implies$ оптимальное решение:\\
    $x_1 = 0$\\
    $x_2 = 6/(-1)$\\
    $x_3 = 0$\\
    $x_4 = 10/1 = 10$\\
    $x_2$ не удовлетворяет условию $x_2 \ge 0$ $\implies$ недопустимое решение\\
    {\bfПоставим вспомогательную задачу:}\\
\end{flushleft}

\begin{equation*}
    \begin{cases}
        x_1 + x_2 - x_3 + y_1= 6 \\
        5x_1 - 10x_2 + x_4 = 10 \\
        x_i \ge 0, i = 1, 2, 3, 4\\
        y_1 \ge 0
    \end{cases}
\end{equation*}
\begin{center}
    $G = - y_1 \rightarrow min$\\
    $G = y_1 \rightarrow max$
\end{center}

\begin{flushleft}
    {\bf1-я симплекс-таблица:}
\end{flushleft}

\begin{center}
    \begin{tabular}{|c | g c c c c c c|} 
         \hline
            Базис & $x_1$ & $x_2$ & $x_3$ & $x_4$ & $y_1$ & $b_i$ & $b_i/$р.с.\\
         \hline
            $y_1$ & 1 & 1 & -1 & 0 & 1 & 6 & 6/1 = 6\\
         \hline
         \rowcolor{LightBlue}
            $x_4$ & \cellcolor{cyan}5 & -10 & 0 & 1 & 0 & 10 & 10/5 = 2\\
         \hline
            $\Delta$ & -1 & -1 & 1 & 0 & 0 & $G = 6$ &\\
         \hline
    \end{tabular}
\end{center}

\begin{flushleft}
    Наибольшая положительная $\Delta$: 1\\
    Разрешающий столбец: $x_1$\\
    Минимальное положительное значение из столбца $b_i$/р.с. : 2\\
    Разрешающий элемент: 5\\
    Не все $\Delta > 0$ $\implies$ решение не оптимально, строим новую таблицу\\
    {\bf2-я симплекс-таблица:}\\
\end{flushleft}

\begin{center}
    \begin{tabular}{|c | c g c c c c c|} 
         \hline
            Базис & $x_1$ & $x_2$ & $x_3$ & $x_4$ & $y_1$ & $b_i$ & $b_i/$р.с.\\
         \hline
         \rowcolor{LightBlue}
            $y_1$ & 0 & \cellcolor{cyan}3 & -1 & -0.2 & 1 & 4 & 4/3 = $\frac{4}{3}$\\
         \hline
            $x_1$ & 5 & -10 & 0 & 1 & 0 & 10 & 10/(-10) = -1\\
         \hline
            $\Delta$ & 0 & -3 & 1 & 0.2 & 0 & $G = 8$ &\\
         \hline
    \end{tabular}
\end{center}

\begin{flushleft}
    Наибольшая положительная $\Delta$: 3\\
    Разрешающий столбец: $x_2$\\
    Минимальное положительное значение из столбца $b_i$/р.с. : $\frac{4}{3}$\\
    Разрешающий элемент: 3\\
    Не все $\Delta > 0$ $\implies$ решение не оптимально, строим новую таблицу\\
    {\bf3-я симплекс-таблица:}\\
\end{flushleft}

\begin{center}
    \begin{tabular}{|c | c c c c c c c|} 
         \hline
            Базис & $x_1$ & $x_2$ & $x_3$ & $x_4$ & $y_1$ & $b_i$ & $b_i/$р.с.\\
         \hline
            $x_2$ & 0 & 3 & -1 & -0.2 & 1 & 4 &\\
         \hline
            $x_1$ & 5 & 0 & -$\frac{10}{3}$ & $\frac{1}{3}$ & $\frac{10}{3}$ & $\frac{70}{3}$ &\\
         \hline
            $\Delta$ & 0 & 0 & 0 & 0 & 1 & $G = 12$ &\\
         \hline
    \end{tabular}
\end{center}

\begin{flushleft}
    Все $\Delta > 0$ $\implies$ решение оптимально\\
    $x_1 = \frac{70}{3} : 5 = \frac{14}{3}$\\
    $x_2 = 4 : 3 = \frac{4}{3}$\\
    $F_{min} = \frac{14}{3} + 2 \cdot \frac{4}{3} = \frac{22}{3}$ 
\end{flushleft}

{\bfОтвет:~} $F_{max}$ не определено; $F_{min} = F(\frac{14}{3}; \frac{4}{3}) = \frac{22}{3}$


\subsubsection{c)}
\begin{equation*}
    F = x_1 + 2x_2 \rightarrow max
\end{equation*}
\begin{equation*}
    \begin{cases}
        2x_1 - x_2 \ge 6 \\
        2x_1 + x_2 \le 1 \\
        x_1 \ge 0, x_2 \ge 0
    \end{cases}
\end{equation*}

\begin{center}
    {\bf
    Решение:}
\end{center}

\begin{flushleft}
Приведем задачу к канонической форме:
\end{flushleft}

\begin{equation*}
    \begin{cases}
        2x_1 - x_2 - x_3 = 6 \\
        2x_1 + x_2 + x_4 = 1 \\
        x_i \ge 0, i = 1, 2, 3, 4
    \end{cases}
\end{equation*}
\begin{center}
    $F - x_1 - 2x_2 = 0$
\end{center}

\begin{flushleft}
    {\bf1-я симплекс-таблица:}\\
\end{flushleft}

\begin{center}
    \begin{tabular}{|c | c g c c c c|} 
         \hline
            Базис & $x_1$ & $x_2$ & $x_3$ & $x_4$ & $b_i$ & $b_i/$р.с.\\
         \hline
            $x_3$ & 2 & -1 & -1 & 0 & 6 & 6/(-1) = -6\\
         \hline
         \rowcolor{LightBlue}
            $x_4$ & 2 & \cellcolor{cyan}1 & 0 & 1 & 1 & 1/1 = 1\\
         \hline
            F(x) & -1 & -2 & 0 & 0 & 0 &\\
         \hline
    \end{tabular}
\end{center}

\begin{flushleft}
    Наименьшее значение в строке F(x): -2\\
    Разрешающий столбец: $x_2$\\
    Минимальное положительное значение из столбца $b_i$/р.с. : 1\\
    Разрешающий элемент: 1\\
    Не все значения в строке F(x) положительные $\implies$ решение не оптимально, строим новую таблицу\\
    {\bf2-я симплекс-таблица:}\\
\end{flushleft}

\begin{center}
    \begin{tabular}{|c | c c c c c c|} 
         \hline
            Базис & $x_1$ & $x_2$ & $x_3$ & $x_4$ & $b_i$ & $b_i/$р.с.\\
         \hline
            $x_3$ & 4 & 0 & -1 & 1 & 7 &\\
         \hline
            $x_2$ & 2 & 1 & 0 & 1 & 1 & \\
         \hline
            F(x) & 3 & 0 & 0 & 2 & 2 &\\
         \hline
    \end{tabular}
\end{center}

\begin{flushleft}
    Все значения в строке F(x) положительные\\
    Оптимальное решение:\\
    $x_1 = 0$\\
    $x_2 = 1/1 = 1$\\
    $x_3 = 7/(-1)$\\
    $x_4 = 0$\\
    $x_3$ не удовлетворяет условию $x_3 \ge 0$ $\implies$ недопустимое решение\\
    {\bfПоставим вспомогательную задачу:}\\
\end{flushleft}

\begin{equation*}
    \begin{cases}
        2x_1 - x_2 - x_3 + y_1= 6 \\
        2x_1 + x_2 + x_4 = 1 \\
        x_i \ge 0, i = 1, 2, 3, 4\\
        y_1 \ge 0
    \end{cases}
\end{equation*}
\begin{center}
    $G = - y_1 \rightarrow max$
\end{center}


\begin{flushleft}
    {\bf1-я симплекс-таблица:}
\end{flushleft}

\begin{center}
    \begin{tabular}{|c | g c c c c c c|} 
         \hline
            Базис & $x_1$ & $x_2$ & $x_3$ & $x_4$ & $y_1$ & $b_i$ & $b_i/$р.с.\\
         \hline
            $y_1$ & 2 & -1 & -1 & 0 & 1 & 6 & 6/(-1) = -6\\
         \hline
         \rowcolor{LightBlue}
            $x_4$ & \cellcolor{cyan}2 & 1 & 0 & 1 & 0 & 1 & 1/1 = 1\\
         \hline
            $\Delta$ & 2 & -1 & -1 & 0 & 0 & $G = -6$ &\\
         \hline
    \end{tabular}
\end{center}

\begin{flushleft}
    Наибольшая положительная $\Delta$: 2\\
    Разрешающий столбец: $x_1$\\
    Минимальное положительное значение из столбца $b_i$/р.с. : 1\\
    Разрешающий элемент: 2\\
    Не все $\Delta < 0$ $\implies$ решение не оптимально, строим новую таблицу\\
    {\bf2-я симплекс-таблица:}\\
\end{flushleft}

\begin{center}
    \begin{tabular}{|c | c c c c c c c|} 
         \hline
            Базис & $x_1$ & $x_2$ & $x_3$ & $x_4$ & $y_1$ & $b_i$ & $b_i/$р.с.\\
         \hline
            $y_1$ & 0 & -2 & -1 & -1 & 1 & 5 & 6/(-1) = -6\\
         \hline
            $x_1$ & 2 & 1 & 0 & 1 & 0 & 1 & 1/1 = 1\\
         \hline
            $\Delta$ & 0 & -2 & -1 & -1 & 0 & $G = -7$ &\\
         \hline
    \end{tabular}
\end{center}

\begin{flushleft}
    Все $\Delta < 0, G < 0 \implies $ в исходной задаче область допустимых значений пустая $\implies F_{max} $ не определено 
\end{flushleft}

{\bfОтвет:~} $F_{max}$ не определено
