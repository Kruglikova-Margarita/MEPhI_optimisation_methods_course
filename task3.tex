\chapter{Экономическая задача}
{\bfВариант 12}
\begin{flushleft}
    Составить математическую модель для прямой и двойственной задачи. Получить решение прямой и двойственной задачи симплекс-методом. Дать экономическую интерпретацию двойственных задач и двойственных оценок.\\
    Для производства четырех видов изделий (А, В, С) предприятие использует три вида сырья: металл, пластмассу, резину. Запасы сырья, технологические коэффициенты (расход каждого вида сырья на производство единицы каждого изделия) представлены в таблице. В ней же указана прибыль от реализации одного изделия каждого вида. Требуется составить такой план выпуска указанных изделий, чтобы обеспечить максимальную прибыль.
\end{flushleft}

\begin{center}
    \begin{tabular}{|c | c | c | c | c | c|} 
         \hline
            & A & B & C & D & запасы\\
         \hline
            металл & 3 & 0 & 1 & 1 & 1000\\
         \hline
            пластмасса & 6 & 4 & 2 & 1 & 1100\\
         \hline
            резина & 9 & 12 & 2 & 5 & 1300\\
         \hline
            прибыль (руб) & 9 & 7 & 2 & 6 & \\
        \hline
    \end{tabular}
\end{center}

\begin{center}
    {\bf
    Решение:}
\end{center}

{\bf1. Постановка задачи}\\
{\bf1.1. Постановка прямой задачи:}
\begin{flushleft}
    Найти оптимальный план производства продукции с максимальной прибылью, для которого достаточно имеющихся ресурсов. $x_1, x_2, x_3, x_4$ – количество произведенной продукции.\\
\end{flushleft}
Целевая функция прямой задачи:
\begin{equation*}
    F = 9x_1 + 7x_2 + 2x_3 + 6x_4 \rightarrow max
\end{equation*}
Ограничения прямой задачи:
\begin{equation*}
    \begin{cases}
        3x_1 + x_3 + x_4 \le 1000 \\
        6x_1 + 4x_2 + 2x_3 + x_4 \le 1100 \\
        9x_1 + 12x_2 + 2x_3 + 5x_4 \le 1300 \\
        x_i \ge 0, i = 1, 2, 3, 4
    \end{cases}
\end{equation*}

{\bf1.2. Постановка двойственной задачи:}
\begin{flushleft}
    Оценить каждый из видов сырья, используемого для производства продукции. Оценки, приписываемые каждому виду сырья, должны быть такими, чтобы оценка всего используемого сырья была минимальна, а суммарная оценка сырья, используемого для производства единицы продукции – не меньше цены единицы продукции.
\end{flushleft}
Целевая функция двойственной задачи:
\begin{equation*}
    G = 1000y_1 + 1100y_2 + 1300y_3 \rightarrow min
\end{equation*}
Ограничения двойственной задачи:
\begin{equation*}
    \begin{cases}
        3y_1 + 6y_2 + 9y_3 \ge 9 \\
        4y_2 + 12y_3 \ge 7 \\
        y_1 + 2y_2 + 2y_3 \ge 2 \\
        y_1 + y_2 + 5y_3 \ge 6 \\
        y_i \ge 0, i = 1, 2, 3
    \end{cases}
\end{equation*}
{\bf2. Решим прямую задачу}, введя 3 фиктивные переменные: $x_5, x_6, x_7$\\
Канонический вид прямой задачи:

\begin{equation*}
    \begin{cases}
        3x_1 + x_3 + x_4 + x_5 = 1000 \\
        6x_1 + 4x_2 + 2x_3 + x_4 + x_6 = 1100 \\
        9x_1 + 12x_2 + 2x_3 + 5x_4 + x_7 = 1300 \\
        x_i \ge 0, i = 1, 2, 3, 4, 5, 6, 7
    \end{cases}
\end{equation*}
\begin{equation*}
    F - 9x_1 - 7x_2 - 2x_3 - 6x_4 = 0
\end{equation*}

\begin{flushleft}
    {\bf1-я симплекс-таблица:}\\
\end{flushleft}

\begin{center}
    \begin{tabular}{|c | g c c c c c c c c|} 
         \hline
            Базис & $x_1$ & $x_2$ & $x_3$ & $x_4$ & $x_5$ & $x_6$ & $x_7$ & $b_i$ & $b_i/$р.с.\\
         \hline
            $x_5$ & 3 & 0 & 1 & 1 & 1 & 0 & 0 & 1000 & $\frac{1000}{3}$\\
         \hline
            $x_6$ & 6 & 4 & 2 & 1 & 0 & 1 & 0 & 1100 & $\frac{550}{3}$ \\
        \hline
        \rowcolor{LightBlue}
            $x_7$ & \cellcolor{cyan}9 & 12 & 2 & 5 & 0 & 0 & 1 & 1300 & $\frac{1300}{9}$ \\
         \hline
            F(x) & -9 & -7 & -2 & -6 & 0 & 0 & 0 & 0 &\\
         \hline
    \end{tabular}
\end{center}

\begin{flushleft}
    Наименьшее значение в строке F(x): -9\\
    Разрешающий столбец: $x_1$\\
    Минимальное положительное значение из столбца $b_i$/р.с. : $\frac{1300}{9}$\\
    Разрешающий элемент: 9\\
    Не все значения в строке F(x) положительные $\implies$ решение не оптимально, строим новую таблицу\\
    {\bf2-я симплекс-таблица:}\\
\end{flushleft}

\begin{center}
    \begin{tabular}{|c | c c c g c c c c c|} 
         \hline
            Базис & $x_1$ & $x_2$ & $x_3$ & $x_4$ & $x_5$ & $x_6$ & $x_7$ & $b_i$ & $b_i/$р.с.\\
         \hline
            $x_5$ & 0 & -4 & $\frac{1}{3}$ & $-\frac{2}{3}$ & 1 & 0 & $-\frac{1}{3}$ & $\frac{1700}{3}$ & -850 \\
         \hline
            $x_6$ & 0 & -4 & $\frac{2}{3}$ & $-\frac{7}{3}$ & 0 & 1 & $-\frac{2}{3}$ & $\frac{700}{3}$ & -100 \\
        \hline
        \rowcolor{LightBlue}
            $x_1$ & 9 & 12 & 2 & \cellcolor{cyan}5 & 0 & 0 & 1 & 1300 & 260 \\
         \hline
            F(x) & 0 & 5 & 0 & -1 & 0 & 0 & 1 & 1300 & -1300\\
         \hline
    \end{tabular}
\end{center}

\begin{flushleft}
    Наименьшее значение в строке F(x): -1\\
    Разрешающий столбец: $x_4$\\
    Минимальное положительное значение из столбца $b_i$/р.с. : 260\\
    Разрешающий элемент: 5\\
    Не все значения в строке F(x) положительные $\implies$ решение не оптимально, строим новую таблицу\\
    {\bf3-я симплекс-таблица:}\\
\end{flushleft}

\begin{center}
    \begin{tabular}{|c | c c c c c c c c c|} 
         \hline
            Базис & $x_1$ & $x_2$ & $x_3$ & $x_4$ & $x_5$ & $x_6$ & $x_7$ & $b_i$ & $C_{B}$\\
         \hline
            $x_5$ & $\frac{6}{5}$ & $-\frac{12}{5}$ & $\frac{3}{5}$ & 0 & 1 & 0 & $-\frac{1}{5}$ & 740 & 0 \\
         \hline
            $x_6$ & $\frac{21}{5}$ & $\frac{8}{5}$ & $\frac{19}{15}$ & 0 & 0 & 1 & $-\frac{1}{5}$ & 840 & 0 \\
        \hline
            $x_4$ & 9 & 12 & 2 & 5 & 0 & 0 & 1 & 1300 & 6 \\
         \hline
            F(x) & $\frac{9}{5}$ & $\frac{37}{5}$ & $\frac{2}{5}$ & 0 & 0 & 0 & $\frac{6}{5}$ & 1560 & \\
         \hline
    \end{tabular}
\end{center}

\begin{flushleft}
    Все значения в строке F(x) положительные $\implies$ решение оптимально\\
    {\bf$x^{*} = (0, 0, 0, 260, 740, 840, 0)$ - оптимальное решение\\
    $F_{max} = F(x^{*}) = 1560$\\
    $y^{*} = (0, 0, \frac{6}{5}, \frac{9}{5}, \frac{37}{5}, \frac{2}{5}, 0)$}\\
\end{flushleft}

{\bf3. Решим двойственную задачу:}\\
{\bf3.1. Решение через условия дополняющей нежесткости.}\\
По 2-й теореме двойственности оптимальное решение двойственной задачи удовлетворяет условиям:
\begin{equation*}
    \begin{cases}
        \displaystyle\sum_{j=1}^{n} (a_{ij} \cdot x_{j}^{*} - b_{i}) \cdot y_{i}^{*} = 0, i = \overline{1, m}\\
        \displaystyle\sum_{i=1}^{m} (a_{ij} \cdot y_{i}^{*} - c_{j}) \cdot x_{j}^{*} = 0, j = \overline{1, n}
    \end{cases}
\end{equation*}
\begin{equation*}
    \begin{cases}
        (3 \cdot 0 + 1 \cdot 0 + 1 \cdot 260 - 1000) \cdot y_{1}^{*} = 0\\
        (6 \cdot 0 + 4 \cdot 0 + 2 \cdot 0 + 1 \cdot 260 - 1100) \cdot y_{2}^{*} = 0\\
        (9 \cdot 0 + 12 \cdot 0 + 2 \cdot 0 + 5 \cdot 260 - 1300) \cdot y_{3}^{*} = 0\\
        (3 \cdot y_{1}^{*} + 6 \cdot y_{2}^{*} + 9 \cdot y_{3}^{*} - 9) \cdot 0 = 0\\
        (4 \cdot y_{2}^{*} + 12 \cdot y_{3}^{*} - 7) \cdot 0 = 0\\
        (1 \cdot y_{1}^{*} + 2 \cdot y_{2}^{*} + 2 \cdot y_{3}^{*} - 2) \cdot 0 = 0\\
        (1 \cdot y_{1}^{*} + 1 \cdot y_{2}^{*} + 5 \cdot y_{3}^{*} - 6) \cdot 260 = 0
    \end{cases}
\end{equation*}
\begin{equation*}
    \begin{cases}
        y_{1}^{*} = 0\\
        y_{2}^{*} = 0\\
        y_{3}^{*} = \frac{6}{5}
    \end{cases}
\end{equation*}
{\bf3.2. Решение по формуле: $y^{*} = C_{B} \cdot A_{B}^{-1}$}
\begin{flushleft}
    $y^{*} = C_{B} \cdot A_{B}^{-1} = (0; 0; 6) \cdot \begin{pmatrix}
1 & 0 & -\frac{1}{5} \\
0 & 1 & -\frac{1}{5}\\
0 & 0 & \frac{1}{5}\\
\end{pmatrix} = (0; 0; \frac{6}{5})$
\end{flushleft}

{\bf4. Анализ результатов}\\
$x_5, x_6, x_7$ - остатки сырья\\
{\bf4.1. Подставим $x^{*}$ в ограничения прямой задачи:}

\begin{equation*}
    \begin{cases}
        3 \cdot 0 + 0 + 260 = 260 \le 1000 \\
        6 \cdot 0 + 4 \cdot 0 + 2 \cdot 0 + 260 = 260 \le 1100 \\
        9 \cdot 0 + 12 \cdot 0 + 2 \cdot 0 + 5 \cdot 260 = 1300 \\
    \end{cases}
\end{equation*}

\begin{flushleft}
    1) 1-е, 2-е условия имеют знак "<", значит 1-й и 2-й ресурсы (металл и пластмасса) не являются дефицитными, их остатки соответственно равны $x^{*}_5 = 740$, $x^{*}_6 = 840$\\
    2) 3-е условие имеет знак "=", значит 3-й ресурс (резина) дефицитный ($x_7 = 0$)
\end{flushleft}
\begin{flushleft}
    Положительную двойственную оценку имеют только дефицитные виды сырья
\end{flushleft}

{\bf4.2. Подставим $y^{*}$ в ограничения двойственной задачи:}

\begin{equation*}
    \begin{cases}
        3 \cdot 0 + 6 \cdot 0 + 9 \cdot \frac{6}{5} = \frac{54}{5} \ge 9 \\
        4 \cdot 0 + 12 \cdot \frac{6}{5} = \frac{72}{5} \ge 7 \\
        0 + 2 \cdot 0 + 2 \cdot \frac{6}{5} = \frac{12}{5} \ge 2 \\
        0 + 0 + 5 \cdot \frac{6}{5} = 6 \\
    \end{cases}
\end{equation*}

\begin{flushleft}
    1) 4-е ограничение имеет знак "=", значит двойственная оценка ресурса, используемого для изготовления изделия вида D в точности равна доходам, а значит изделия производить целесообразно\\
    2) 1-е, 2-е, 3-е ограничения имеют знак ">", значит производить изделия A, B, C нецелесообразно
\end{flushleft}

{\bf4.3.}
    Величина двойственных оценок показывает, насколько возрастет значение целевой функции при увеличении запасов дефицитного ресурса на одну единицу.\\
    Увеличение запасов 3-го ресурса (резина) на единицу приведет к новому оптимальному плану: $F_{max} = 1561.2$
    Коэффициент $A_{b}^{-1}$ показывают, что увеличение прибыли достигается засчет увеличения выпуска продукции D на 0.2 единицы; при этом запасы металла и пластмассы сократятся на 0.2 единиц соответственно 

{\bf5. Анализ устойчивости двойственных оценок}
\newline
Определим интервалы устойчивости:\\
$x_{b new}^{*} = x_{b} + A_{b}^{-1} \cdot \Delta b = A_{b}^{-1} \cdot (b + \Delta b)$\\
$A_{b}^{-1} \cdot (b + \Delta b) \ge 0$

$A_{b}^{-1} \cdot (b + \Delta b) = \begin{pmatrix}
1 & 0 & -\frac{1}{5} \\
0 & 1 & -\frac{1}{5}\\
0 & 0 & \frac{1}{5}\\
\end{pmatrix}
\begin{pmatrix}
1000 + \Delta b_1 \\
1100 + \Delta b_2 \\
1300 + \Delta b_3 \\
\end{pmatrix} = \begin{pmatrix}
740 + \Delta b_1 - \frac{1}{5} \Delta b_3 \\
840 + \Delta b_2 - \frac{1}{5} \Delta b_3 \\
260 + \frac{1}{5} \Delta b_3 \\
\end{pmatrix} \ge 0$\\
Частные случаи:\\
1) $\Delta b_2 = \Delta b_3 = 0 \implies \Delta b_1 \ge -740 \implies$ запасы 1-го ресурса можно уменьшать не более чем на 740 единиц, при этом оптимальный план двойственной задачи не изменится.\\
2) $\Delta b_1 = \Delta b_3 = 0 \implies \Delta b_2 \ge -840 \implies$ запасы 2-го ресурса можно уменьшать не более чем на 840 единиц, при этом оптимальный план двойственной задачи не изменится.\\
3) $\Delta b_1 = \Delta b_2 = 0 \implies \Delta b_3 \in [-1300; 3700] \implies$ при увеличении запасов 3-го ресурса не более чем на 3700 единиц и уменьшении его запасов не более чем на 1300 единиц значение целевой функции не изменится.\\

Предположим, что $\Delta b_2 = -100, \Delta b_3 = -200$. Тогда:\\
$\begin{pmatrix}
x_5^{new}\\
x_6^{new}\\
x_4^{new}\\
\end{pmatrix} = \begin{pmatrix}
740 + \frac{1}{5} \cdot 200 \\
840 - 100 + \frac{1}{5} \cdot 200 \\
260 - \frac{1}{5} \cdot 200 \\
\end{pmatrix} \ge 0 \implies \Delta b_2, \Delta b_3$ сохраняют оценки ресурсов в пределах устойчивости.
$x_{new} = (0; 0; 0; 220; 780; 780; 0)$\\
Целевая функция изменится на 240 и станет равной $F_{new} = 6 \cdot 220 = 1320$
