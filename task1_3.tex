\chapter{Двойственная задача}

{\bf Вариант №1\\ Для "а" составить и решить геометрически и симплекс – методом задачу двойственную данной.}

\begin{center}
    {\bf
    Решение:}
\end{center}

\subsubsection{1) Нахождение минимума двойственной задачи}

\begin{flushleft}
Прямая задача на максимум:
\end{flushleft}

\begin{equation*}
    F = x_1 + x_2 \rightarrow max
\end{equation*}
\begin{equation*}
    \begin{cases}
        x_1 + 2x_2 \le 8 \\
        6x_1 - x_2 \le 3 \\
        x_1 \ge 0, x_2 \ge 0
    \end{cases}
\end{equation*}

\begin{flushleft}
Из пункта 1а $\implies x_{max} = (\frac{14}{13}; \frac{45}{13}), F_{max} = \frac{59}{13}$
\end{flushleft}

\begin{flushleft}
Поставим двойственную задачу:
\end{flushleft}

\begin{center}
    $F^{*} = 8y_1 + 3y_2 \rightarrow min$
\end{center}
\begin{equation*}
    \begin{cases}
        y_1 + 6y_2 \ge 1\\
        2y_1 - y_2 \ge 1\\
        y_i \ge 0, i = 1, 2\\
    \end{cases}
\end{equation*}

\subsubsection{1.1) Решение с помощью теоремы 2}

\begin{equation*}
    \begin{cases}
        (1 \cdot x_1^{*} + 2 \cdot x_2^{*} - 8)y_1^{*} = 0\\
        (6 \cdot x_1^{*} - x_2^{*} - 3)y_2^{*} = 0\\
        (y_1^{*} + 6y_2^{*} - 1)x_1^{*} = 0\\
        (2y_1^{*} - y_2^{*} - 1)x_2^{*} = 0\\
    \end{cases}
\end{equation*}

\begin{equation*}
    \begin{cases}
        (1 \cdot \frac{14}{13} + 2 \cdot \frac{45}{13} - 8)y_1^{*} = 0\\
        (6 \cdot \frac{14}{13} - \frac{45}{13} - 3)y_2^{*} = 0\\
        (y_1^{*} + 6y_2^{*} - 1) \cdot \frac{14}{13} = 0\\
        (2y_1^{*} - y_2^{*} - 1) \cdot \frac{45}{13} = 0\\
    \end{cases}
\end{equation*}

\begin{equation*}
    \begin{cases}
        0 \cdot y_1^{*} = 0\\
        0 \cdot y_2^{*} = 0\\
        y_1^{*} + 6y_2^{*} - 1 = 0\\
        2y_1^{*} - y_2^{*} - 1 = 0\\
    \end{cases}
\end{equation*}

\begin{equation*}
    \begin{cases}
        y_1^{*} > 0\\
        y_2^{*} > 0\\
        y_1^{*} + 6y_2^{*} - 1 = 0\\
        2y_1^{*} - y_2^{*} - 1 = 0\\
    \end{cases}
\end{equation*}

\begin{flushleft}
В результате решения системы получим:
\end{flushleft}
\begin{equation*}
    \begin{cases}
        y_1^{*} = \frac{7}{13}\\
        y_2^{*} = \frac{1}{13}\\
    \end{cases}
\end{equation*}

\begin{flushleft}
$F_{min}^{*} = F^{*}(\frac{7}{13}; \frac{1}{13}) = 8 \cdot \frac{7}{13} + 3 \cdot \frac{1}{13} = \frac{59}{13}$
\end{flushleft}

{\bfОтвет:~} $F_{min}^{*} = F^{*}(\frac{7}{13}; \frac{1}{13}) = \frac{59}{13} = F_{max}$

\subsubsection{1.2) Решение с помощью теоремы 3}

\begin{flushleft}
В пункте 2а была получена итоговая симплекс-таблица:    
\end{flushleft}

\begin{center}
    \begin{tabular}{|c | c c c c c|} 
         \hline
            Базис & $x_1$ & $x_2$ & $x_3$ & $x_4$ & $b_i$\\
         \hline
            $x_2$ & 0 & 2 & $\frac{12}{13}$ & $-\frac{2}{13}$ & $\frac{90}{13}$\\
         \hline
            $x_1$ & 6.5 & 0 & 0.5 & 1 & 7\\
         \hline
            F(x) & 0 & 0 & $\frac{7}{13}$ & 1 & $\frac{59}{13}$\\
         \hline
    \end{tabular}
\end{center}

\begin{flushleft}
Поделим строку, соответствующую $x_2$, на 2; строку, соответствующую $x_1$, на 6.5. Добавим в таблицу столбец $C_{B}$,значениями которого являются коэффициенты при соответствующих $x_i$ в F(x):
\end{flushleft}

\begin{center}
    \begin{tabular}{|c | c c c c c c|} 
         \hline
            Базис & $x_1$ & $x_2$ & $x_3$ & $x_4$ & $b_i$ & $C_{B}$\\
         \hline
            $x_2$ & 0 & 1 & $\frac{6}{13}$ & $-\frac{1}{13}$ & $\frac{45}{13}$ & 1\\
         \hline
            $x_1$ & 1 & 0 & $\frac{1}{13}$ & $\frac{2}{13}$ & $\frac{14}{13}$ & 1\\
         \hline
            F(x) & 0 & 0 & $\frac{7}{13}$ & 1 & $\frac{59}{13}$ &\\
         \hline
    \end{tabular}
\end{center}

\begin{flushleft}
$y^{*} = C_{B} \cdot A_{B}^{-1} = (1; 1) \cdot \begin{pmatrix}
\frac{6}{13} & -\frac{1}{13} \\
\frac{1}{13} & \frac{2}{13} \\
\end{pmatrix} = (\frac{6}{13} + \frac{1}{13}; -\frac{1}{13} + \frac{2}{13}) = (\frac{7}{13}; \frac{1}{13})$ \\
$F_{min}^{*} = F^{*}(\frac{7}{13}; \frac{1}{13}) = 8 \cdot \frac{7}{13} + 3 \cdot \frac{1}{13} = \frac{59}{13}$
\end{flushleft}

{\bfОтвет:~} $F_{min}^{*} = F^{*}(\frac{7}{13}; \frac{1}{13}) = \frac{59}{13} = F_{max}$



\subsubsection{2) Нахождение максимума двойственной задачи}

\begin{flushleft}
Прямая задача на минимум:
\end{flushleft}

\begin{equation*}
    F = x_1 + x_2 \rightarrow min
\end{equation*}
\begin{equation*}
    \begin{cases}
        x_1 + 2x_2 \le 8 \\
        6x_1 - x_2 \le 3 \\
        x_1 \ge 0, x_2 \ge 0
    \end{cases}
\end{equation*}

\begin{flushleft}
Из пункта 1а $\implies x_{min} = (0; 0), F_{min} = 0$
\end{flushleft}

\begin{flushleft}
Поставим двойственную задачу:
\end{flushleft}

\begin{center}
    $F^{*} = 8y_1 + 3y_2 \rightarrow max$
\end{center}
\begin{equation*}
    \begin{cases}
        y_1 + 6y_2 \ge 1\\
        2y_1 - y_2 \ge 1\\
        y_i \ge 0, i = 1, 2\\
    \end{cases}
\end{equation*}

\subsubsection{2.1) Решение с помощью теоремы 2}

\begin{equation*}
    \begin{cases}
        (1 \cdot x_1^{*} + 2 \cdot x_2^{*} - 8)y_1^{*} = 0\\
        (6 \cdot x_1^{*} - x_2^{*} - 3)y_2^{*} = 0\\
        (y_1^{*} + 6y_2^{*} - 1)x_1^{*} = 0\\
        (2y_1^{*} - y_2^{*} - 1)x_2^{*} = 0\\
    \end{cases}
\end{equation*}

\begin{equation*}
    \begin{cases}
        (1 \cdot 0 + 2 \cdot 0 - 8)y_1^{*} = 0\\
        (6 \cdot 0 - 0 - 3)y_2^{*} = 0\\
        (y_1^{*} + 6y_2^{*} - 1) \cdot 0 = 0\\
        (2y_1^{*} - y_2^{*} - 1) \cdot 0 = 0\\
    \end{cases}
\end{equation*}

\begin{equation*}
    \begin{cases}
        y_1^{*} = 0\\
        y_2^{*} = 0\\
    \end{cases}
\end{equation*}

\begin{flushleft}
$F_{max}^{*} = F^{*}(0; 0) = 8 \cdot 0 + 3 \cdot 0 = 0$
\end{flushleft}

{\bfОтвет:~} $F_{max}^{*} = F^{*}(0; 0) = 0 = F_{min}$

\subsubsection{2.2) Решение с помощью теоремы 3}

\begin{flushleft}
Первая симплекс-таблица в пункте 2а является итоговой симплекс-таблицей для прямой задачи на минимум:
\end{flushleft}

\begin{center}
    \begin{tabular}{|c | c c c c c|} 
         \hline
            Базис & $x_1$ & $x_2$ & $x_3$ & $x_4$ & $b_i$\\
         \hline
            $x_3$ & 1 & 2 & 1 & 0 & 8\\
         \hline
            $x_4$ & 6 & -1 & 0 & 1 & 3\\
         \hline
            F(x) & -1 & -1 & 0 & 0 & 0\\
         \hline
    \end{tabular}
\end{center}

\begin{flushleft}
Добавим в таблицу столбец $C_{B}$, значениями которого являются коэффициенты при соответствующих $x_i$ в F(x):
\end{flushleft}

\begin{center}
    \begin{tabular}{|c | c c c c c c|} 
         \hline
            Базис & $x_1$ & $x_2$ & $x_3$ & $x_4$ & $b_i$ & $C_{B}$\\
         \hline
            $x_3$ & 1 & 2 & 1 & 0 & 8 & 0\\
         \hline
            $x_4$ & 6 & -1 & 0 & 1 & 3 & 0\\
         \hline
            F(x) & -1 & -1 & 0 & 0 & 0 &\\
         \hline
    \end{tabular}
\end{center}

\begin{flushleft}
$y^{*} = C_{B} \cdot A_{B}^{-1} = (0; 0) \cdot \begin{pmatrix}
1 & 0 \\
0 & 1 \\
\end{pmatrix} = (0; 0)$ \\
$F_{min}^{*} = F^{*}(0; 0) = 8 \cdot 0 + 3 \cdot 0 = 0$
\end{flushleft}

{\bfОтвет:~} $F_{max}^{*} = F^{*}(0; 0) = 0 = F_{min}$
